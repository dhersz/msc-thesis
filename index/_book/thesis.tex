\newlength{\cslhangindent}
\setlength{\cslhangindent}{1.5em}
\newenvironment{CSLReferences}%
  {}%
  {\par}

  \documentclass[msc,numbers]{coppe}

\usepackage{amsmath,amssymb}
\usepackage{hyperref}
\usepackage{longtable}
\usepackage{booktabs}

\providecommand{\tightlist}{%
  \setlength{\itemsep}{0pt}\setlength{\parskip}{0pt}}

\makelosymbols
\makeloabbreviations

\begin{document}

  \title{O Impacto da Tarifa do Transporte Público sobre a Acessibilidade}
  \foreigntitle{The Impact of Transit Fare on Accessibility}
    \author{Daniel}{Herszenhut Meirelles Santos}
  

    \advisor{Prof.}{Licinio}{da Silva Portugal}{D.Sc.}
    \advisor{Prof.}{Matheus Henrique}{de Sousa Oliveira}{D.Sc.}
  

    \examiner{Prof.}{Nome Completo do Primeiro Examinador}{D.Sc.}
    \examiner{Prof.}{Nome Completo do Segundo Examinador}{Ph.D}
    \examiner{Prof.}{Nome Completo do Terceiro Examinador}{Ph.D}
    \department{PET}
  \date{05}{2021}

    \keyword{Acessibilidade}
    \keyword{Custo monetário}
    \keyword{Equidade}
  
  \maketitle

  \frontmatter
  \dedication{``O suburbano quando chega atrasado\\
O patrão mal-humorado\\
Diz que mora logo ali\\
Mas é porque não anda nesse trem lotado\\
Com o peito amargurado\\
Baldeando por aí\\
Imagine quem vem lá de Japeri\\
Imagine quem vem lá de Japeri''\\
(Guará e Jorginho das Rosas, 1984)}

    \chapter*{Agradecimentos}
  Gostaria de agradecer a X, Y e Z
 \begin{abstract}
blablablabla
 \end{abstract}
  \begin{foreignabstract}
blablablablablabl
  \end{foreignabstract}
  \tableofcontents

  \printlosymbols
  \printloabbreviations

  \mainmatter

  \hypertarget{intro}{%
  \chapter{Introdução}\label{intro}}

  À acessibilidade são frequentemente atribuídos diferentes significados, a depender da forma como é operacionalizada e do contexto em que é analisada (\protect\hyperlink{ref-geurs2004accessibility}{Geurs and van Wee 2004}; \protect\hyperlink{ref-kwan1998spacetime}{Kwan 1998}; \protect\hyperlink{ref-vanwee2016accessible}{van Wee 2016}). Em linhas gerais, no entanto, pode se dizer que a acessibilidade é determinada pela distribuição espacial de atividades, ou de potenciais oportunidades, e pela facilidade pelas quais elas podem ser acessadas (\protect\hyperlink{ref-handy1997measuring}{Handy and Niemeier 1997}).

  Estudos recentes levam ao entendimento de que maiores níveis de acessibilidade se traduzem em maiores potenciais de realização e satisfação pessoal (\protect\hyperlink{ref-martens2012justice}{Martens 2012}), e de que políticas de transporte devem conferir níveis mínimos de acessibilidade a determinados destinos e reduzir desigualdades de acesso a oportunidades (\protect\hyperlink{ref-pereira2017distributive}{Pereira, Schwanen, and Banister 2017}). A acessibilidade, portanto, vem cada vez mais sendo considerada um objetivo de políticas de transporte nos últimos anos (\protect\hyperlink{ref-neutens2010equity}{Neutens et al. 2010}; \protect\hyperlink{ref-vanwee2016accessible}{van Wee 2016}), o que tem levado pesquisadores e tomadores de decisão a investigar como essas políticas afetam de forma distinta a acessibilidade de diferentes localidades e grupos socioeconômicos (\protect\hyperlink{ref-lucas2016method}{Lucas, van Wee, and Maat 2016}).

  As estimativas dos níveis de acesso a oportunidades de um indivíduo ou local podem ser feitas através de diversas medidas, sendo estas classificadas em diferentes grupos, conforme a perspectiva que adotam ao medir a acessibilidade (\protect\hyperlink{ref-geurs2004accessibility}{Geurs and van Wee 2004}). Dentre esses grupos, um dos mais frequentemente utilizados no planejamento urbano e de transportes é o de medidas baseadas em localidades, que se diferencia dos demais ao associar a acessibilidade a um lugar, e não a um indivíduo ou atividade (\protect\hyperlink{ref-miller2007placebased}{Miller 2007}).

  As medidas desse grupo atribuem a cada deslocamento um fator de impedância que pode ser expresso em forma de tempo, distância, dinheiro ou alguma outra função de desutilidade. Via de regra, no entanto, a maioria dos estudos que usam essas medidas estima a acessibilidade considerando apenas os custos de tempo de viagem (\protect\hyperlink{ref-el-geneidy2016cost}{El-Geneidy et al. 2016}; \protect\hyperlink{ref-venter2016assessing}{Venter 2016}), embora idealmente todos os custos associados a um deslocamento devam ser adequadamente representados, como, por exemplo, os relacionados ao conforto, à conveniência e ao valor monetário de uma viagem (\protect\hyperlink{ref-bocarejo2014innovative}{Bocarejo et al. 2014}; \protect\hyperlink{ref-dalvi1976measurement}{Dalvi and Martin 1976}).

  O custo monetário, em particular, é especialmente relevante para análises de políticas de transporte: primeiramente, porque muitas pessoas têm dificuldade de arcar com os custos dos deslocamentos cotidianos nos grandes certos urbanos, sejam estes tidos como desenvolvidos ou em desenvolvimento (\protect\hyperlink{ref-venter2011transport}{Venter 2011}); em segundo lugar, mais especificamente para análises relacionadas ao transporte público, porque o preço de uma viagem pode variar conforme fatores como a linha, o modo de transporte e o operador do serviço, não sendo, portanto, linearmente correlacionado ao usualmente representado tempo de viagem (\protect\hyperlink{ref-venter2016assessing}{Venter 2016}); e porque, no caso em que serviços similares são oferecidos a preço distintos, indivíduos de baixa renda podem acabar relegados a viagens mais baratas e potencialmente menos confiáveis (\protect\hyperlink{ref-conway2019getting}{Conway and Stewart 2019}).

  Consequentemente, custos monetários vêm recebendo cada vez mais atenção na literatura de acessibilidade (e.g. \protect\hyperlink{ref-guzman2018accessibility}{Guzman and Oviedo 2018}; \protect\hyperlink{ref-liu2020measuring}{Liu and Kwan 2020}; \protect\hyperlink{ref-oviedo2019bus}{Oviedo et al. 2019}). A maior parte dos estudos que consideram esses custos se concentra em como introduzi-los a medidas de acessibilidade para posteriormente diagnosticar desigualdades relacionadas aos transportes (e.g. \protect\hyperlink{ref-bocarejo2014innovative}{Bocarejo et al. 2014}; \protect\hyperlink{ref-el-geneidy2016cost}{El-Geneidy et al. 2016}; \protect\hyperlink{ref-ma2017modeling}{Ma, Masoud, and Idris 2017}). Até o momento, no entanto, nenhum trabalho investigou de que forma a incorporação do custo monetário a medidas de acessibilidade afeta os resultados e conclusões derivadas de análises de acessibilidade e equidade nos transportes; ou seja, como os resultados de análises que consideram este custo se diferenciam daqueles que provêm de análises que não o consideram. Este trabalho visa preencher essa lacuna.

  Esta dissertação se apoia sobre um estudo de caso que analisa a distribuição da acessibilidade ao emprego por transporte público na cidade do Rio de Janeiro. O Rio tem recebido, recentemente, bastante atenção de pesquisadores preocupados com questões relacionadas à equidade e à justiça em seus sistemas de transportes e uso do solo (e.g. \protect\hyperlink{ref-barboza2021balancing}{Barboza et al. 2021}; \protect\hyperlink{ref-carneiro2019espraiamento}{Carneiro et al. 2019}; \protect\hyperlink{ref-pereira2018transport}{Pereira 2018}; \protect\hyperlink{ref-pereira2019distributional}{Pereira et al. 2019}). Ainda assim, até então nenhum trabalho que tem a cidade como plano de fundo levou em consideração restrições monetárias em suas estimativas de acessibilidade. A introdução do custo monetário ao fator de impedância de uma medida de acessibilidade neste contexto, portanto, potencialmente destaca e revela desigualdades de acesso a oportunidades até então não identificadas em estudos prévios.

  \hypertarget{problema-de-pesquisa-e-hipuxf3teses}{%
  \section{Problema de pesquisa e hipóteses}\label{problema-de-pesquisa-e-hipuxf3teses}}

  Frente ao contexto apresentado, surge a seguinte pergunta, a partir da qual se elabora o problema de pesquisa:
  \begin{itemize}
  \tightlist
  \item
    A incorporação do custo monetário a medidas de acessibilidade afeta os resultados e conclusões derivadas de análises de acessibilidade e equidade nos transportes? Caso positivo, como?
  \end{itemize}
  A partir deste problema, tem-se como hipótese básica:
  \begin{itemize}
  \tightlist
  \item
    Por se tratar de um custo não necessariamente correlacionado com elementos normalmente incorporados ao fator de impedância de medidas de acessibilidade (principalmente o tempo de viagem), a incorporação do custo monetário a estas medidas afeta os resultados e conclusões derivadas de análises de acessibilidade e equidade nos transportes.
  \end{itemize}
  De forma semelhante, tem-se como hipótese secundária:
  \begin{itemize}
  \tightlist
  \item
    A incorporação do custo monetário a medidas de acessibilidade não necessariamente torna a distribuição da acessibilidade mais ou menos equitativa; como o custo monetário afeta a distribuição da acessibilidade depende em larga escala de elementos como políticas tarifárias, características operacionais e espaciais da rede de transporte público, e da co-distribuição espacial da população e das oportunidades.
  \end{itemize}
  \hypertarget{objetivos}{%
  \section{Objetivos}\label{objetivos}}

  Alinhado ao problema destacado, configura-se como objetivo primário da pesquisa:

  Os objetivos secundários, complementares ao primário, são:

  \hypertarget{estrutura-da-dissertauxe7uxe3o}{%
  \section{Estrutura da dissertação}\label{estrutura-da-dissertauxe7uxe3o}}

  \hypertarget{references}{%
  \chapter*{References}\label{references}}
  \addcontentsline{toc}{chapter}{References}

  \bibliographystyle{$biblio-style$}
  \bibliography{thesis}

  \hypertarget{refs}{}
  \begin{CSLReferences}{1}{0}
  \leavevmode\hypertarget{ref-barboza2021balancing}{}%
  Barboza, Matheus H. C., Mariana S. Carneiro, Claudio Falavigna, Gregório Luz, and Romulo Orrico. 2021. {``Balancing Time: {Using} a New Accessibility Measure in {Rio} de {Janeiro}.''} \emph{Journal of Transport Geography} 90 (January): 102924. \url{https://doi.org/10.1016/j.jtrangeo.2020.102924}.

  \leavevmode\hypertarget{ref-bocarejo2014innovative}{}%
  Bocarejo, Juan Pablo, Ingrid Joanna Portilla, Juan Miguel Velásquez, Mónica Natalia Cruz, Andrés Peña, and Daniel Ricardo Oviedo. 2014. {``An Innovative Transit System and Its Impact on Low Income Users: The Case of the {Metrocable} in {Medellín}.''} \emph{Journal of Transport Geography} 39 (July): 49--61. \url{https://doi.org/10.1016/j.jtrangeo.2014.06.018}.

  \leavevmode\hypertarget{ref-carneiro2019espraiamento}{}%
  Carneiro, Mariana, Juliana Toledo, Marcelino Aurélio, and Romulo Orrico. 2019. {``Espraiamento Urbano e Exclusão Social. {Uma} análise Da Acessibilidade Dos Moradores Da Cidade Do {Rio} de {Janeiro} Ao Mercado de Trabalho.''} \emph{EURE (Santiago)} 45 (136): 51--70. \url{https://doi.org/10.4067/S0250-71612019000300051}.

  \leavevmode\hypertarget{ref-conway2019getting}{}%
  Conway, Matthew Wigginton, and Anson F. Stewart. 2019. {``Getting {Charlie} Off the {MTA}: A Multiobjective Optimization Method to Account for Cost Constraints in Public Transit Accessibility Metrics.''} \emph{International Journal of Geographical Information Science} 33 (9): 1759--87. \url{https://doi.org/10.1080/13658816.2019.1605075}.

  \leavevmode\hypertarget{ref-dalvi1976measurement}{}%
  Dalvi, M. Q., and K. M. Martin. 1976. {``The Measurement of Accessibility: {Some} Preliminary Results.''} \emph{Transportation} 5 (March): 17--42. \url{https://doi.org/10.1007/BF00165245}.

  \leavevmode\hypertarget{ref-el-geneidy2016cost}{}%
  El-Geneidy, Ahmed, David Levinson, Ehab Diab, Genevieve Boisjoly, David Verbich, and Charis Loong. 2016. {``The Cost of Equity: {Assessing} Transit Accessibility and Social Disparity Using Total Travel Cost.''} \emph{Transportation Research Part A: Policy and Practice} 91 (September): 302--16. \url{https://doi.org/10.1016/j.tra.2016.07.003}.

  \leavevmode\hypertarget{ref-geurs2004accessibility}{}%
  Geurs, Karst T., and Bert van Wee. 2004. {``Accessibility Evaluation of Land-Use and Transport Strategies: Review and Research Directions.''} \emph{Journal of Transport Geography} 12 (2): 127--40. \url{https://doi.org/10.1016/j.jtrangeo.2003.10.005}.

  \leavevmode\hypertarget{ref-guzman2018accessibility}{}%
  Guzman, Luis A., and Daniel Oviedo. 2018. {``Accessibility, Affordability and Equity: {Assessing} {`Pro-Poor'} Public Transport Subsidies in {Bogotá}.''} \emph{Transport Policy} 68 (September): 37--51. \url{https://doi.org/10.1016/j.tranpol.2018.04.012}.

  \leavevmode\hypertarget{ref-handy1997measuring}{}%
  Handy, S L, and D A Niemeier. 1997. {``Measuring {Accessibility}: {An Exploration} of {Issues} and {Alternatives}.''} \emph{Environment and Planning A: Economy and Space} 29 (7): 1175--94. \url{https://doi.org/10.1068/a291175}.

  \leavevmode\hypertarget{ref-kwan1998spacetime}{}%
  Kwan, Mei-Po. 1998. {``Space-{Time} and {Integral Measures} of {Individual Accessibility}: {A Comparative Analysis Using} a {Point}-Based {Framework}.''} \emph{Geographical Analysis} 30 (3): 191--216. \url{https://doi.org/10.1111/j.1538-4632.1998.tb00396.x}.

  \leavevmode\hypertarget{ref-liu2020measuring}{}%
  Liu, Dong, and Mei-Po Kwan. 2020. {``Measuring {Job Accessibility Through Integrating Travel Time}, {Transit Fare And Income}: {A Study Of The Chicago Metropolitan Area}.''} \emph{Tijdschrift Voor Economische En Sociale Geografie} 111 (4): 671--85. \url{https://doi.org/10.1111/tesg.12415}.

  \leavevmode\hypertarget{ref-lucas2016method}{}%
  Lucas, Karen, Bert van Wee, and Kees Maat. 2016. {``A Method to Evaluate Equitable Accessibility: Combining Ethical Theories and Accessibility-Based Approaches.''} \emph{Transportation} 43 (3): 473--90. \url{https://doi.org/10.1007/s11116-015-9585-2}.

  \leavevmode\hypertarget{ref-ma2017modeling}{}%
  Ma, Zhenyuan (Eric), Abdul Rahman Masoud, and Ahmed O. Idris. 2017. {``Modeling the {Impact} of {Transit Fare Change} on {Passengers}' {Accessibility}.''} \emph{Transportation Research Record: Journal of the Transportation Research Board} 2652 (1): 78--86. \url{https://doi.org/10.3141/2652-09}.

  \leavevmode\hypertarget{ref-martens2012justice}{}%
  Martens, Karel. 2012. {``Justice in Transport as Justice in Accessibility: Applying {Walzer}'s {`{Spheres} of {Justice}'} to the Transport Sector.''} \emph{Transportation} 39 (6): 1035--53. \url{https://doi.org/10.1007/s11116-012-9388-7}.

  \leavevmode\hypertarget{ref-miller2007placebased}{}%
  Miller, Harvey. 2007. {``Place-{Based} Versus {People}-{Based Geographic Information Science}.''} \emph{Geography Compass} 1 (3): 503--35. \url{https://doi.org/10.1111/j.1749-8198.2007.00025.x}.

  \leavevmode\hypertarget{ref-neutens2010equity}{}%
  Neutens, Tijs, Tim Schwanen, Frank Witlox, and Philippe De Maeyer. 2010. {``Equity of {Urban Service Delivery}: {A Comparison} of {Different Accessibility Measures}.''} \emph{Environment and Planning A: Economy and Space} 42 (7): 1613--35. \url{https://doi.org/10.1068/a4230}.

  \leavevmode\hypertarget{ref-oviedo2019bus}{}%
  Oviedo, Daniel, Lynn Scholl, Marco Innao, and Lauramaria Pedraza. 2019. {``Do {Bus Rapid Transit Systems Improve Accessibility} to {Job Opportunities} for the {Poor}? {The Case} of {Lima}, {Peru}.''} \emph{Sustainability} 11 (10): 2795. \url{https://doi.org/10.3390/su11102795}.

  \leavevmode\hypertarget{ref-pereira2018transport}{}%
  Pereira, Rafael H. M. 2018. {``Transport Legacy of Mega-Events and the Redistribution of Accessibility to Urban Destinations.''} \emph{Cities} 81 (November): 45--60. \url{https://doi.org/10.1016/j.cities.2018.03.013}.

  \leavevmode\hypertarget{ref-pereira2019distributional}{}%
  Pereira, Rafael H. M., David Banister, Tim Schwanen, and Nate Wessel. 2019. {``Distributional Effects of Transport Policies on Inequalities in Access to Opportunities in {Rio} de {Janeiro}.''} \emph{Journal of Transport and Land Use} 12 (1). \url{https://doi.org/10.5198/jtlu.2019.1523}.

  \leavevmode\hypertarget{ref-pereira2017distributive}{}%
  Pereira, Rafael H. M., Tim Schwanen, and David Banister. 2017. {``Distributive Justice and Equity in Transportation.''} \emph{Transport Reviews} 37 (2): 170--91. \url{https://doi.org/10.1080/01441647.2016.1257660}.

  \leavevmode\hypertarget{ref-vanwee2016accessible}{}%
  van Wee, Bert. 2016. {``Accessible Accessibility Research Challenges.''} \emph{Journal of Transport Geography} 51 (February): 9--16. \url{https://doi.org/10.1016/j.jtrangeo.2015.10.018}.

  \leavevmode\hypertarget{ref-venter2011transport}{}%
  Venter, Christoffel. 2011. {``Transport Expenditure and Affordability: {The} Cost of Being Mobile.''} \emph{Development Southern Africa} 28 (1): 121--40. \url{https://doi.org/10.1080/0376835X.2011.545174}.

  \leavevmode\hypertarget{ref-venter2016assessing}{}%
  ---------. 2016. {``Assessing the Potential of Bus Rapid Transit-Led Network Restructuring for Enhancing Affordable Access to Employment {} {The} Case of {Johannesburg}'s {Corridors} of {Freedom}.''} \emph{Research in Transportation Economics} 59 (November): 441--49. \url{https://doi.org/10.1016/j.retrec.2016.05.006}.

  \end{CSLReferences}
\end{document}
%% 
%%
%% End of file `example.tex'.
