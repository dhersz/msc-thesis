
  \documentclass[msc,numbers]{coppe}

\usepackage{amsmath,amssymb}
\usepackage{hyperref}
\usepackage{longtable}
\usepackage{booktabs}

\providecommand{\tightlist}{%
  \setlength{\itemsep}{0pt}\setlength{\parskip}{0pt}}

\makelosymbols
\makeloabbreviations

\begin{document}

  \title{O Impacto da Tarifa do Transporte Público sobre a Acessibilidade}
  \foreigntitle{The Impact of Transit Fare on Accessibility}
    \author{Daniel}{Herszenhut Meirelles Santos}
  

    \advisor{Prof.}{Licinio}{da Silva Portugal}{D.Sc.}
    \advisor{Prof.}{Matheus Henrique}{de Sousa Oliveira}{D.Sc.}
  

    \examiner{Prof.}{Nome Completo do Primeiro Examinador}{D.Sc.}
    \examiner{Prof.}{Nome Completo do Segundo Examinador}{Ph.D}
    \examiner{Prof.}{Nome Completo do Terceiro Examinador}{Ph.D}
    \department{PET}
  \date{05}{2021}

    \keyword{Acessibilidade}
    \keyword{Custo monetário}
    \keyword{Equidade}
  
  \maketitle

  \frontmatter
  \dedication{``O suburbano quando chega atrasado\\
O patrão mal-humorado\\
Diz que mora logo ali\\
Mas é porque não anda nesse trem lotado\\
Com o peito amargurado\\
Baldeando por aí\\
Imagine quem vem lá de Japeri\\
Imagine quem vem lá de Japeri''\\
(Guará e Jorginho das Rosas, 1984)}

    \chapter*{Agradecimentos}
  Gostaria de agradecer a X, Y e Z
 \begin{abstract}
blablablabla
 \end{abstract}
  \begin{foreignabstract}
blablablablablabl
  \end{foreignabstract}
  \tableofcontents

  \printlosymbols
  \printloabbreviations

  \mainmatter

  \hypertarget{intro}{%
  \chapter{Introdução}\label{intro}}

  Placeholder

  \hypertarget{problema-de-pesquisa-e-hipuxf3teses}{%
  \section{Problema de pesquisa e hipóteses}\label{problema-de-pesquisa-e-hipuxf3teses}}

  \hypertarget{objetivos}{%
  \section{Objetivos}\label{objetivos}}

  \hypertarget{estrutura-da-dissertauxe7uxe3o}{%
  \section{Estrutura da dissertação}\label{estrutura-da-dissertauxe7uxe3o}}

  \hypertarget{revisuxe3o-da-literatura}{%
  \chapter{Revisão da literatura}\label{revisuxe3o-da-literatura}}

  \hypertarget{acessibilidade-e-equidade}{%
  \section{Acessibilidade e equidade}\label{acessibilidade-e-equidade}}

  Discussões sobre a equidade de sistemas e serviços de transportes permeiam, hoje, o mundo do planejamento urbano e de transportes. Neste contexto, entende-se que a equidade está relacionada à justiça distributiva de benefícios e malefícios de políticas de transporte entre os membros de uma sociedade (DI CIOMMO, SHIFTAN, 2017, PEREIRA, SCHWANEN, et al., 2017). A partir desta definição podem ser identificados três componentes-chave da equidade nos transportes: os custos e benefícios distribuídos; os princípios que determinam se uma distribuição é justa e socialmente aceita; e os grupos populacionais para quais os benefícios são distribuídos (DI CIOMMO, SHIFTAN, 2017).

  Estudos recentes buscam determinar quais benefícios de políticas de transporte devem ser considerados em análises de equidade. Análises de custo-benefício, por exemplo, frequentemente empregam o valor monetário de reduções de tempo de viagem na avaliação de projetos (LUCAS, VAN WEE, et al., 2016, MARTENS, DI CIOMMO, 2017). Seus resultados, no entanto, estão sujeitos a vieses que tendem a exacerbar as desigualdades relacionadas aos transportes, uma vez que os ganhos monetários associados a viagens mais curtas são calculados a partir da renda dos indivíduos, consequentemente avaliando de forma mais positiva projetos que beneficiem as classes mais abastadas (GOODWIN, 1974, MARTENS, DI CIOMMO, 2017).

  Outros trabalhos utilizam os padrões de viagem para descrever e diagnosticar desigualdades, sob a justificativa de que eles descrevem os níveis de participação social praticados na sociedade (PEREIRA, SCHWANEN, et al., 2017). Não há como definir, no entanto, se esses padrões são determinados por escolhas voluntárias ou por condições às quais os indivíduos são involuntariamente expostos, e nem como precisar se as necessidades particulares de cada indivíduo são satisfeitas pelo conjunto de deslocamentos realizados (MARTENS, 2019, PEREIRA, SCHWANEN, et al., 2017).

  Como alternativa, portanto, diversos autores sugerem que os efeitos distributivos de políticas de transporte devem ser avaliados pelos níveis de acessibilidade que elas promovem (DI CIOMMO, SHIFTAN, 2017, LUCAS, VAN WEE, et al., 2016, MARTENS, 2012, MARTENS, DI CIOMMO, 2017, PEREIRA, SCHWANEN, et al., 2017, VAN WEE, GEURS, 2011), argumentando que o principal objetivo dos sistemas de transportes é promover o acesso a atividades valorizadas pela população (VAN WEE, 2016). A importância da acessibilidade não está apenas no fato de que uma pessoa a ``exerce'' ao participar de uma atividade, mas também no fato de que esta pessoa pode optar por esta atividade dentre um conjunto maior de oportunidades (MARTENS, 2019). A acessibilidade, portanto, está ligada à noção de liberdade de escolha e a maiores níveis de satisfação pessoal (MARTENS, 2012), e é condição necessária, embora não suficiente, para a promoção de uma sociedade mais equitativa (VAN WEE, GEURS, 2011).

  Como essa acessibilidade deve ser, então, distribuída entre a população? A resposta varia conforme os princípios adotados para julgar uma dada distribuição como justa, o que leva pesquisadores a investigar sobre quais teorias de justiça devem se apoiar análises de transportes e acessibilidade (PEREIRA, SCHWANEN, et al., 2017). Três teorias se mostram particularmente relevantes para essas análises: utilitarismo, igualitarismo e suficientarismo (VAN WEE, GEURS, 2011).

  O utilitarismo estabelece que uma política é moralmente justa quando a soma total de seus benefícios, habitualmente calculada a partir do valor monetário de economias de tempo de viagem, é maior do que a de seus custos, e portanto está relacionado às já mencionadas análises de custo-benefício (MARTENS, 2019, VAN WEE, GEURS, 2011). Análises utilitárias, portanto, não valorizam a acessibilidade como um fim, mas sim a utilidade, como percebida pelos indivíduos, das viagens que os conectam a atividades (PEREIRA, SCHWANEN, et al., 2017). Esta perspectiva, no entanto, não é adequada para análises de equidade, pois estimam a utilidade de uma dada política segundo a propensão dos indivíduos a pagar por deslocamentos que os permitam alcançar atividades que valorizem: políticas que tragam maiores retornos às classes mais altas são vistas como benéficas, por maximizarem a utilidade total ao privilegiar aqueles que estão mais propensos a pagar por economias de tempo de viagem e cujo tempo é tido como mais valioso do que o de seus concidadãos de classes mais baixas (MARTENS, 2019, PEREIRA, SCHWANEN, et al., 2017, VAN WEE, GEURS, 2011).

  Uma perspectiva mais promissora se baseia no igualitarismo (LUCAS, VAN WEE, et al., 2016, PEREIRA, SCHWANEN, et al., 2017). Esta teoria, conforme formulação adaptada ao contexto do planejamento de transportes, valoriza a acessibilidade em si como um bem primário que deve ser distribuído de forma a diminuir desigualdades de oportunidades (PEREIRA, SCHWANEN, et al., 2017, VAN WEE, GEURS, 2011). Sob esta perspectiva, portanto, uma política é considerada justa caso ela aumente os níveis de acessibilidade de populações vulneráveis e, consequentemente, diminua a desigualdade de acesso a oportunidades entre diferentes grupos socioeconômicos (LUCAS, VAN WEE, et al., 2016).

  O igualitarismo, no entanto, foca apenas na diferença de acesso a oportunidades entre diferentes indivíduos e grupos, e não nos níveis absolutos de acessibilidade que cada indivíduo detém. O suficientarismo, por sua vez, estabelece que todos devem ter condições mínimas de acessibilidade a determinadas atividades essenciais, e políticas justas, portanto, garantem que essas condições sejam satisfeitas (LUCAS, VAN WEE, et al., 2016). Embora ainda não haja um método amplamente utilizado na definição desses patamares mínimos, pode-se afirmar que sua identificação depende de um conjunto de características econômicas, sociais e culturais próprias a cada área de estudo (PEREIRA, SCHWANEN, et al., 2017).

  O suficientarismo, dessa forma, é apresentado como uma perspectiva complementar à igualitarista, o que nos permite formular que políticas de transporte devem visar a diminuição da desigualdade de acesso a oportunidades entre grupos socioeconômicos distintos, e, além disso, também garantir que indivíduos vulneráveis tenham condições básicas de acesso a atividades essenciais (PEREIRA, SCHWANEN, et al., 2017, VAN WEE, 2016). Esta formulação compreende os três componentes-chaves da equidade dos transportes, como caracterizada por Di Ciommo e Shiftan (2017): os benefícios a serem distribuídos (a acessibilidade); os princípios que determinam como distribuí-los (o igualitarismo e o suficientarismo); e os grupos populacionais que devem ser privilegiados (grupos socioeconomicamente vulneráveis e privados de acesso a atividades essenciais). Ela, contudo, não especifica como devem ser estimados os níveis de acessibilidade, ficando à cargo dos pesquisadores determinar quais medidas são mais adequadas a cada situação.

  \hypertarget{medidas-de-acessibilidade}{%
  \section{Medidas de acessibilidade}\label{medidas-de-acessibilidade}}

  Medidas de acessibilidade são extensivamente utilizadas por pesquisadores, agências de transporte e tomadores de decisão como ferramentas para avaliar, a partir da distribuição espacial de atividades, o quão equitativo é o acesso a serviços e bens públicos (NEUTENS, SCHWANEN, et al., 2010). A busca por estimativas de acessibilidade facilmente comunicáveis, pouco computacionalmente intensivas e metodologicamente robustas, no entanto, levou ao desenvolvimento de um sem-número de medidas (PÁEZ, SCOTT, et al., 2012). Consequentemente, diversos estudos se empenharam em classificá-las segundo diferentes critérios, apresentando também suas vantagens, desvantagens e características individuais (e.g.~GEURS, VAN WEE, 2004, HANDY, NIEMEIER, 1997, NEUTENS, SCHWANEN, et al., 2010, PÁEZ, SCOTT, et al., 2012).

  A classificação mais frequentemente empregada na literatura separa essas medidas em quatro diferentes grupos: as baseadas em infraestrutura, em utilidade, em pessoas e em localidades (GEURS, VAN WEE, 2004). Medidas baseadas em infraestrutura avaliam a performance da infraestrutura de transporte existente e são frequentemente expressas pelo nível de serviço de um determinado equipamento, porém ignoram os efeitos do uso do solo sobre a acessibilidade. Essas medidas, portanto, não são apropriadas para análises de equidade (GEURS, VAN WEE, 2004, MARTENS, GOLUB, 2012).

  Medidas baseadas em utilidade, por sua vez, interpretam a acessibilidade como a satisfação que um indivíduo deriva de um deslocamento a uma determinada atividade de desejo. Esta satisfação varia conforme, entre outros fatores, o modo utilizado, o horário em que a viagem é realizada e as características socioeconômicas do indivíduo que realiza esse deslocamento, e é expressa em termos monetários para que os componentes anteriormente citados possam ser comparados e agregados em uma mesma medida (NIEMEIER, 1997). Uma das principais vantagens desse grupo de medidas é que elas são teoricamente robustas - ou seja, são sensíveis a características dos sistemas de transportes e uso do solo, a variações temporais na oferta de serviços e oportunidades e a características particulares de cada indivíduo (GEURS, VAN WEE, 2004). No entanto, são de difícil operacionalização, por necessitarem de uma grande quantidade de dados desagregados a níveis individuais, e de também difícil comunicação e interpretação, por se basearem em modelos econométricos e estatísticos que precisam ser previamente explicados a planejadores e tomadores de decisão (e.g.~DE JONG, DALY, et al., 2007).

  Adicionalmente, trabalhos recentes argumentam que medidas baseadas em utilidade não são adequadas para análises de equidade nos transportes. Como apresentado na seção anterior, uma abordagem igualitarista ao problema da desigualdade nos transportes implica em valorizar a acessibilidade em si como um bem a ser distribuído entre a população, e não a utilidade que resulta do acesso a determinadas oportunidades (PEREIRA, SCHWANEN, et al., 2017, VAN WEE, GEURS, 2011). Ainda, o enfoque na satisfação que um indivíduo obtém ao alcançar uma determinada oportunidade traz à tona dois problemas. Primeiramente, pode-se argumentar que a satisfação derivada de uma viagem que torna menos atrativo, ou mais perigoso, o deslocamento de pessoas mais vulneráveis não deveria ser contabilizada (MARTENS, GOLUB, 2012); por exemplo, políticas que privilegiam usuários de transporte individual motorizado ao aumentar limites de velocidade, e que põem em risco usuários de transporte ativo e público, não devem ser vistas como benéficas, mesmo que levem a níveis médios de satisfação maior do que outras alternativas. Também deve-se levar em consideração que pessoas estão sujeitas a diferentes condições de vida, e que isso pode alterar o nível de satisfação que derivam de viagens que se dão em condições similares: uma pessoa de alta renda, por exemplo, acostumada a deslocamentos em automóveis particulares, pode obter menor satisfação ao realizar uma viagem de transporte público do que seus concidadãos de baixa renda, já acostumados a modos coletivos - isso não quer dizer, no entanto, que políticas de transporte, buscando maximizar a utilidade total da sociedade, devam prover melhores condições de acesso ao transporte, público ou privado, para aqueles já historicamente favorecidos (MARTENS, 2019, MARTENS, GOLUB, 2012).

  As duas categorias de medidas de acessibilidade mais amplamente utilizadas em estudos de equidades nos transportes, consequentemente, são as baseadas em pessoas e localidades. Medidas baseadas em pessoas se caracterizam por estimar a acessibilidade a partir da perspectiva de indivíduos que estão sujeitos a diferentes restrições espaciais, temporais e pessoais ao longo de um dia (KWAN, 1998, MILLER, 2007). Essas medidas usam prismas de espaço-tempo para descrever os conjuntos de atividades que podem ser realizadas dentro do ``orçamento temporal'' de cada pessoa (HÄGERSTRAAND, 1970), que pode variar conforme seu gênero, raça, ocupação, etc.

  A principal vantagem deste grupo de medidas está em sua robustez teórica, que provém da estimativa dos níveis de acessibilidade de forma altamente desagregada e que, consequentemente, as torna altamente sensíveis a características particulares de cada indivíduo (GEURS, VAN WEE, 2004). Esta abordagem desagregada, no entanto, requer uma grande quantidade de dados e implica em operações computacionalmente muito intensivas, o que acaba restringindo aplicações dessas medidas a regiões e parcelas da população relativamente pequenas (e.g.~KWAN, 1998, NEUTENS, SCHWANEN, et al., 2010).

  Medidas baseadas em localidades, por sua vez, associam a acessibilidade a um lugar, e não a uma pessoa (MILLER, 2007). Dessa forma, precisam de dados menos desagregados (a nível de unidade espacial, e não de indivíduo), requerem menor capacidade computacional e são de mais simples comunicação e interpretação do que medidas baseadas em pessoas, o que as torna muito populares entre estudos de planejamento urbano e de transportes (GEURS, VAN WEE, 2004).

  Graças a essas qualidades, uma medida baseada em localidades será utilizada no estudo de caso conduzido nesta dissertação. Análises que dependam dessas medidas, no entanto, incorrem na chamada falácia ecológica (PEREIRA, BANISTER, et al., 2019) - ou seja, partem do pressuposto que todos os indivíduos dentro de uma mesma unidade espacial possuem as mesmas características, como, por exemplo, o nível de renda. Embora possa ser minimizado com o advento de grades espaciais de alta resolução, este problema está fundamentalmente associado a esse tipo de medidas. Consequentemente, elas são tidas como teoricamente menos robustas do que as baseadas em pessoas (GEURS, VAN WEE, 2004).

  Medidas baseadas em localidades associam a cada deslocamento um custo, usualmente expresso pelo tempo de viagem, e dois grandes subgrupos dessas medidas se diferenciam pela forma como definem a função de desutilidade que caracteriza este custo. Medidas de oportunidades cumulativas estimam quantas oportunidades podem ser alcançadas a partir de uma origem por viagens que não superem um valor limite de custo, e se destacam por serem de especialmente fácil operacionalização, comunicação e interpretação (GEURS, VAN WEE, 2004). Este valor limite, no entanto, é frequentemente estabelecido de forma arbitrária e pouco criteriosa, podendo impactar de forma significativa estimativas de acessibilidade e análises de equidade que delas derivem (PEREIRA, 2019).

  Ainda, medidas de oportunidades cumulativas estabelecem que todas as oportunidades são igualmente alcançáveis e desejadas, desde que acessíveis a custos menores do que o limite. Medidas potenciais gravitacionais, por sua vez, estabelecem que quanto maior o custo para acessar uma oportunidade, menor é a acessibilidade que dela se obtém (HANDY, NIEMEIER, 1997). Nesse caso, a função utilizada para representar a impedância de uma viagem pode assumir diferentes formas, sendo algumas das mais comuns as de decaimento exponencial, logístico e gaussiano.

  A principal vantagem dessas medidas frente às de oportunidades cumulativas está no fato de incorporarem a percepção de que oportunidades que exigem grandes custos para serem alcançadas são menos acessíveis do que as que exigem menores custos para tal (GEURS, VAN WEE, 2004). Por outro lado, são mais difíceis de comunicar e interpretar, porque seu resultado não representa o total de oportunidades que pode ser alcançado, e sim o somatório de oportunidades ponderadas pela função de impedância (ibid). Adicionalmente, como essas funções são compostas por parâmetros que precisam ser calibrados para o contexto em que são aplicadas, medidas potenciais gravitacionais também requerem maior disponibilidade de dados do que as de oportunidades cumulativas (HANDY, NIEMEIER, 1997). E, ainda, como o método utilizado nesta calibração pode variar tanto pela forma da função de decaimento quanto pelos dados disponíveis, trabalhos que utilizam essas medidas são mais difíceis de serem replicados em contextos diferentes do que o de sua aplicação original, o que pode levar a tomadas de decisão e escolhas metodológicas arbitrárias (PEREIRA, 2019).

  Uma terceira categoria de medidas baseadas em localidades é a de medidas competitivas. Essas medidas não apenas consideram quantas oportunidades podem ser alcançadas a partir de uma origem, mas também quantas pessoas conseguem acessar cada uma dessas oportunidades e, consequentemente, competir pelo seu uso (SHEN, 1998). São, portanto, mais adequadas para estimar os níveis de acessibilidade a atividades que possuem limitações de capacidade, como empregos e matrículas escolares, por exemplo, mas não são tão populares quanto as duas categorias anteriormente descritas por serem de mais difícil operacionalização e, principalmente, comunicação (GEURS, VAN WEE, 2004).

  Independente da categoria a que pertençam, medidas baseadas em localidades são frequentemente operacionalizadas de forma que apenas o tempo de viagem seja representado em suas funções de impedância (EL-GENEIDY, LEVINSON, et al., 2016). Ainda assim, muitos trabalhos advogam que estimativas de acessibilidade devem idealmente levar em consideração outras características de um deslocamento, como seu valor monetário, condições de conforto, previsibilidade, entre outras (BOCAREJO, PORTILLA, et al., 2014, DALVI, MARTIN, 1976, HANDY, NIEMEIER, 1997), e vem crescendo o número de trabalhos que incorporam custos monetários a medidas baseadas em localidades. A próxima seção apresenta uma revisão desses trabalhos, relacionando-os a alguns critérios destacados até aqui, como robustez teórica e facilidade de operacionalização, comunicação e interpretação.

  \hypertarget{o-custo-monetuxe1rio-em-medidas-baseadas-em-localidades}{%
  \section{O custo monetário em medidas baseadas em localidades}\label{o-custo-monetuxe1rio-em-medidas-baseadas-em-localidades}}

  Medidas baseadas em localidades são algumas das medidas de acessibilidade mais populares entre estudos de planejamento urbano e de transportes (GEURS, VAN WEE, 2004). A maior parte dos trabalhos que as utilizam, no entanto, representa o custo de um deslocamento apenas pela sua duração, ignorando outros fatores que também podem se apresentar como obstáculos para o pleno acesso a oportunidades, como o valor monetário, o conforto e a segurança de uma viagem, por exemplo (BOCAREJO, PORTILLA, et al., 2014, EL-GENEIDY, LEVINSON, et al., 2016, VENTER, 2016).

  O custo monetário é especialmente relevante para análises de sistemas de transporte público, pois o preço de uma viagem pode variar conforme a linha utilizada, o modo de transporte e o operador do serviço, entre outros fatores. Questões relacionadas à equidade desses sistemas surgem quando, por exemplo, a tarifa de uma determinado modo o torna proibitivo a pessoas de baixa renda, ou quando diversas linhas oferecem serviços similares a preços diferentes, potencialmente relegando indivíduos pobres a deslocamentos mais baratos, porém menos confortáveis, confiáveis ou seguros (CONWAY, STEWART, 2019). Este custo, ainda, não é necessariamente correlacionado ao tempo de viagem, impactando análises de equidade de maneira possivelmente imprevisível (VENTER, 2016).

  Adicionalmente, incorporar custos monetários a uma medida de acessibilidade aumenta sua sensibilidade a necessidades e características particulares de cada indivíduo (como o orçamento destinado ao transporte) e a mudanças no sistema de transportes (pois linhas e modos também podem ser descritos segundo os custos associadas a eles), dois dos critérios teóricos que medidas de acessibilidade devem buscar seguir, de acordo com Geurs e van Wee (2004). Este ponto é particularmente relevante para medidas baseadas em localidades, que, como apresentado na seção anterior, são tidas como menos teoricamente robustas do que medidas baseadas em pessoas.

  Esses são alguns dos argumentos levantados por trabalhos que incorporam tais custos a medidas baseadas em localidades. A maior parte desses estudos olha para como que o custo monetário deve ser introduzido a essas medidas e para como que esses custos ajudam a diagnosticar desigualdades em termos de acesso a oportunidades (BOCAREJO, PORTILLA, et al., 2014, BOCAREJO S., OVIEDO H., 2012, EL-GENEIDY, LEVINSON, et al., 2016, GUZMAN, OVIEDO, 2018, GUZMAN, OVIEDO, et al., 2017, LIU, KWAN, 2020, MA, MASOUD, et al., 2017, OVIEDO, SCHOLL, et al., 2019, RODRIGUEZ, PERALTA-QUIRÓS, et al., 2017, VAN DIJK, KRYGSMAN, et al., 2015, VENTER, 2016). Outros também analisam a relação entre acessibilidade e a percepção de bem-estar em diferentes esferas da vida (LIONJANGA, VENTER, 2018) e propõem um algoritmo de roteamento multimodal que leva em consideração restrições de cunho monetário (CONWAY, STEWART, 2019). Cabe notar que a maioria desses trabalhos se concentra em cidades do Sul Global, especialmente na América do Sul e na África do Sul, provavelmente porque o debate relacionado às imposições econômicas e financeiras do transporte público sobre populações vulneráveis possui mais destaque neste contexto do que em países do Norte Global. A despeito da crescente preocupação com o efeito de restrições monetárias sobre análises de acessibilidade, até então nenhum estudo investigou como ignorar custos monetários em estimativas de acessibilidade pode enviesar o resultado de análises de equidade e conclusões que delas derivem. Este trabalho busca preencher esta lacuna.

  Os trabalhos que introduziram custos monetários a medidas baseadas em localidades também se distinguem quanto à abordagem utilizada para operacionalizar as medidas. A maior parte deles faz uso de uma função de custo generalizado, em que um valor monetário é associado ao tempo para agregar tempo e dinheiro em um único custo total. Este valor do tempo (VDT) pode ser definido como constante para toda a área de estudo (EL-GENEIDY, LEVINSON, et al., 2016, MA, MASOUD, et al., 2017) ou como uma função de características socioeconômicas que podem variar espacialmente (OVIEDO, SCHOLL, et al., 2019, VENTER, 2016).

  Funções de custo generalizado, no entanto, apresentam algumas desvantagens. Em primeiro lugar, a agregação de tempo e dinheiro em um mesmo custo torna a interpretação e comunicação de resultados mais complexa. Também não há um método padrão de como estimar este VDT, o que leva a tomadas de decisão arbitrárias e prejudica a comparação dos resultados de diferentes estudos. Mais importante, contudo, é o fato de que estimativas de VDT estão sujeitas a limitações que podem torná-las inapropriadas para o uso em análises de equidade (MARTENS, DI CIOMMO, 2017). Alguns podem argumentar, por exemplo, que um VDT fixo produz estimativas de acessibilidade pouco precisas, por não levar em consideração diferenças na percepção do valor do tempo entre diferentes grupos socioeconômicos. Por outro lado, a utilização de VDTs que variam conforme a renda do indivíduo ou segundo sua disposição a pagar por economias de tempo de viagem implica no pressuposto problemático de que o tempo de pessoas ricas vale mais do que o de pessoas pobres. Neste caso, avaliações de equidade podem favorecer grupos historicamente privilegiados (GOODWIN, 1974).

  Uma alternativa às funções de custo generalizado consiste no uso de duas funções de impedância distintas, uma relacionada ao tempo de viagem e outra ao custo monetário. Conway e Stewart (2019) e Rodriguez et al.~(2017) utilizam esta abordagem ao estimar os níveis de acessibilidade usando medidas de oportunidades cumulativas em que são estabelecidos valores limites tanto para a duração quanto para o valor monetário dos deslocamentos. Neste caso, limites de custo monetário podem ser definidos, de forma similar a como hoje são estabelecidos limites de tempo de viagem, de acordo com gastos médios com transportes na região ou segundo valores tidos como aceitáveis, dadas as restrições orçamentárias da população. Esta forma de incorporar restrições de custo monetário a medidas de acessibilidade é relativamente nova e tem recebido menos atenção do que sua alternativa até o momento, porém pode se mostrar vantajosa porque evita o problema de determinar um VDT justo, e preserva a fácil comunicação e interpretação dos resultados. Por estas razões, esta abordagem será utilizada no estudo de caso descrito nos pŕoximos capítulos.

  \hypertarget{references}{%
  \chapter*{References}\label{references}}
  \addcontentsline{toc}{chapter}{References}

  \bibliographystyle{$biblio-style$}
  \bibliography{thesis}

\end{document}
%% 
%%
%% End of file `example.tex'.
