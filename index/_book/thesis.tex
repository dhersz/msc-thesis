\newlength{\cslhangindent}
\setlength{\cslhangindent}{1.5em}
\newenvironment{CSLReferences}%
  {}%
  {\par}

  \documentclass[msc,numbers]{coppe}

\usepackage{amsmath,amssymb}
\usepackage{hyperref}
\usepackage{longtable}
\usepackage{booktabs}

\providecommand{\tightlist}{%
  \setlength{\itemsep}{0pt}\setlength{\parskip}{0pt}}

\makelosymbols
\makeloabbreviations

\begin{document}

  \title{O Impacto da Tarifa do Transporte Público sobre a Acessibilidade}
  \foreigntitle{The Impact of Transit Fare on Accessibility}
    \author{Daniel}{Herszenhut Meirelles Santos}
  

    \advisor{Prof.}{Licinio}{da Silva Portugal}{D.Sc.}
    \advisor{Prof.}{Matheus Henrique}{de Sousa Oliveira}{D.Sc.}
  

    \examiner{Prof.}{Nome Completo do Primeiro Examinador}{D.Sc.}
    \examiner{Prof.}{Nome Completo do Segundo Examinador}{Ph.D}
    \examiner{Prof.}{Nome Completo do Terceiro Examinador}{Ph.D}
    \department{PET}
  \date{05}{2021}

    \keyword{Acessibilidade}
    \keyword{Custo monetário}
    \keyword{Equidade}
  
  \maketitle

  \frontmatter
  \dedication{A alguém cujo valor é digno desta dedicatória.}

    \chapter*{Agradecimentos}
  Gostaria de agradecer a X, Y e Z
 \begin{abstract}
blablablabla
 \end{abstract}
  \begin{foreignabstract}
blablablablablabl
  \end{foreignabstract}
  \tableofcontents
  \listoffigures

  \listoftables
  \printlosymbols
  \printloabbreviations

  \mainmatter

  \hypertarget{intro}{%
  \chapter{Introdução}\label{intro}}

  À acessibilidade são frequentemente atribuídos diferentes significados, a depender da forma como é operacionalizada e do contexto em que é analisada (\protect\hyperlink{ref-geurs2004accessibility}{Geurs and van Wee 2004}). Em linhas gerais, pode se dizer que a acessibilidade é determinada pela distribuição espacial de atividades, ou de potenciais oportunidades, e pela facilidade pelas quais elas podem ser acessadas (\protect\hyperlink{ref-handy1997measuring}{Handy and Niemeier 1997}).

  O entendimento de que maiores níveis de acessibilidade indicam maiores potenciais de realização e satisfação pessoal (\protect\hyperlink{ref-martens2012justice}{Martens 2012}), e de que políticas de transporte devem conferir níveis mínimos de acessibilidade a determinados destinados e reduzir desigualdades de acesso a oportunidades (\protect\hyperlink{ref-pereira2017distributive}{Pereira, Schwanen, and Banister 2017}), leva pesquisadores e tomadores de decisão a investigar como tais políticas afetam diferentes grupos socioeconômicas e regiões (\protect\hyperlink{ref-lucas2016method}{Lucas, van Wee, and Maat 2016}).

  Consequentemente, A acessibilidade vem cada vez mais sendo considerada um objetivo de políticas de transporte nos últimos anos (\protect\hyperlink{ref-neutens2010equity}{Neutens et al. 2010}).

  medidas de acessibilidade

  Accessibility has been increasingly considered as a key transport policy goal in recent years. The understanding that higher accessibility levels indicates higher potential for personal fulfillment and satisfaction (Martens, 2012) and that transport policies should consider minimum accessibility standards for key destinations and reduce inequalities of opportunities (Pereira et al., 2017) leads researchers and decision-makers to investigate how such policies affect different regions and social groups (Lucas et al., 2016). Most studies measure accessibility considering solely travel time impedance, ignoring other elements that might hinder access to activities, such as monetary costs (Bocarejo et al., 2014; El-Geneidy et al., 2016; Venter, 2016). These costs are especially relevant to transit-related analyses, since the price of a public transport trip might vary according to factors such as route, travel mode and service operator, and therefore is not linearly correlated to travel time (Venter, 2016).

  Monetary costs are gradually receiving more attention in the accessibility literature (e.g.~Guzman \& Oviedo, 2018; Liu \& Kwan, 2020; Oviedo et al., 2019). However, there is still little understanding about whether incorporating monetary costs into accessibility measurements can affect the conclusions and policy recommendations derived from transport equity analyses. This paper examines how and to what extent simultaneously incorporating travel time and monetary costs into accessibility estimates may impact transport equity assessments, looking at employment accessibility by transit in the city of Rio de Janeiro, Brazil. Rio has recently received growing attention by researchers investigating accessibility and equity issues (e.g.~Barboza et al., 2021; Carneiro et al., 2019; Pereira, 2018; Pereira, Banister, et al., 2019). Nonetheless, previous examinations have only taken into account travel time impedance, ignoring monetary costs and its constraint effects. In this study accessibility levels were calculated with a cumulative opportunity measure using threshold values both for travel time and monetary cost simultaneously. Analyses were conducted considering different combinations of time and cost thresholds to comprehend how accessibility estimates are affected by the interplay of these variables.

  The coming sections are structured as follows: Section 2 presents an overview of recent studies that incorporate monetary costs into accessibility measures; Section 3 exhibits a brief description of Rio's transport system and some of its socioeconomic variables; Section 4 details the data and methods used in the research; Section 5 presents the main results and a discussion on those; finally, Section 6 presents the main conclusions drawn from the results and some recommendations for future studies.

  \hypertarget{references}{%
  \chapter*{References}\label{references}}
  \addcontentsline{toc}{chapter}{References}

  \bibliographystyle{$biblio-style$}
  \bibliography{thesis}

  \hypertarget{refs}{}
  \begin{CSLReferences}{1}{0}
  \leavevmode\hypertarget{ref-geurs2004accessibility}{}%
  Geurs, Karst T., and Bert van Wee. 2004. {``Accessibility Evaluation of Land-Use and Transport Strategies: Review and Research Directions.''} \emph{Journal of Transport Geography} 12 (2): 127--40. \url{https://doi.org/10.1016/j.jtrangeo.2003.10.005}.

  \leavevmode\hypertarget{ref-handy1997measuring}{}%
  Handy, S L, and D A Niemeier. 1997. {``Measuring {Accessibility}: {An Exploration} of {Issues} and {Alternatives}.''} \emph{Environment and Planning A: Economy and Space} 29 (7): 1175--94. \url{https://doi.org/10.1068/a291175}.

  \leavevmode\hypertarget{ref-lucas2016method}{}%
  Lucas, Karen, Bert van Wee, and Kees Maat. 2016. {``A Method to Evaluate Equitable Accessibility: Combining Ethical Theories and Accessibility-Based Approaches.''} \emph{Transportation} 43 (3): 473--90. \url{https://doi.org/10.1007/s11116-015-9585-2}.

  \leavevmode\hypertarget{ref-martens2012justice}{}%
  Martens, Karel. 2012. {``Justice in Transport as Justice in Accessibility: Applying {Walzer}'s {`{Spheres} of {Justice}'} to the Transport Sector.''} \emph{Transportation} 39 (6): 1035--53. \url{https://doi.org/10.1007/s11116-012-9388-7}.

  \leavevmode\hypertarget{ref-neutens2010equity}{}%
  Neutens, Tijs, Tim Schwanen, Frank Witlox, and Philippe De Maeyer. 2010. {``Equity of {Urban Service Delivery}: {A Comparison} of {Different Accessibility Measures}.''} \emph{Environment and Planning A: Economy and Space} 42 (7): 1613--35. \url{https://doi.org/10.1068/a4230}.

  \leavevmode\hypertarget{ref-pereira2017distributive}{}%
  Pereira, Rafael H. M., Tim Schwanen, and David Banister. 2017. {``Distributive Justice and Equity in Transportation.''} \emph{Transport Reviews} 37 (2): 170--91. \url{https://doi.org/10.1080/01441647.2016.1257660}.

  \end{CSLReferences}
\end{document}
%% 
%%
%% End of file `example.tex'.
